\documentclass[11pt,a4paper]{article}
\usepackage{graphicx}
\usepackage{amssymb, amsmath}
\usepackage{url}
\usepackage{polski}
\usepackage{subfigure}
\usepackage[utf8]{inputenc} 

\title{Wprowadzenie do eksploracji danych tekstowych w sieci WWW\\ Sprawozdanie nr 1.\\ \large Ocena narzędzi bioinformatycznych}
\author{Piotr Jastrzębski\\ Piotr Król \\ Rafał Karolewski}
\date{}
\begin{document}
\maketitle

\section{Szczegółowy opis zadania}\label{opis}
Należy opracować program, który dla zadanej listy linków do narzędzi bioinformatycznych oceni ich jakość np.~poprzez analizę cytowań danego narzędzia (najpierw należy znaleźć artykuły, które opisują dane narzędzie).

\section{Załozenia projektu}

Wejście - lista narzędzi bioinformatycznych w~formie tekstowej\\
Wyjście - lista danych narzędzi posortowana według jakości wraz z~wynikiem funkcji oceny

Wyniki funkcji oceny będące miarą jakości zadanych narzędzi wyliczany będzie według następujących kryteriów:
\begin{itemize}
	\item \textbf{Liczba artykułów, w~których wystąpiło odniesienie do danego narzędzia}
	\item \textbf{Liczba cytowań}
	\item \textbf{\emph{h-indeks} autora źródła, w~którym wystąpiło odniesienie do narzędzia }
	\item \textbf{\emph{h-indeks} autorów cytowań}
	\item \textbf{liczba wyników zwracanych dla zapytania Google ,,[narzędzie] bioinformatics''}
	\item \textbf{liczba artykułów na Wikipedii odnoszących się do narzędzia}
\end{itemize}

Ogólny współczynnik jakości wyliczony zostanie jako średnia ważona wszystkich kryteriów z~odpowiednio dobranymi wagami. Wektor wag zostanie wyznaczony na podstawie ważności poszczególnych kryteriów.

\section{Narzędzia do wydobywania wiedzy}

\begin{itemize}
\item \textbf{pubMed} --  internetowa baza danych obejmująca artykuły z~dziedziny medycyny i~nauk biologicznych \cite{pubmed}
\item \textbf{Google Scholar} \cite{scholar}
\item \textbf{Google API} \cite{googleApi}
\item \textbf{MediaWiki API} \cite{mediaWiki}
\item \textbf{web2py} \cite{web2py}
\end{itemize}

\section{Implementacja}
Ze względu na uniwersalność planujemy użycie języka Python. Doskonale sprawdzi się on w~przetwarzaniu tekstów i pozyskiwaniu danych ze stron. Interfejs użytkownika zrealizowany zostanie w formie webowej przy użyciu {H}{T}{M}{L} albo w~formie aplikacji okienkowej Qt.

\subsection{Wczytanie listy narzędzi}
Lista narzędzi zostanie wczytana z odpowiednich pól tekstowych poprzez przeglądarkowy interfejs aplikacji albo z pliku zapisanego w formacie CSV. Interfejs webowy będzie zawierał pola do wpisania listy narzędzi oraz przycisk rozpoczynający przetwarzanie informacji.

\subsection{Zwrócenie wyników}
Wyniki rankingowania narzędzi zostaną zwrócone w formie tabeli wyświetlonej na stronie wygenerowanej przez aplikację. Tabela ta będzie zawierała dwie kolumny: nazwę narzędzia oraz liczbę uzyskanych punktów rankingowych. Rekordy będą posortowane w kolejności malejącej względem liczby punktów.

\begin{thebibliography}{9}
\bibitem{scholar}
  \emph{Google Scholar}
  \url{http://scholar.google.pl/}

\bibitem{pubMed}
  \emph{PubMed}
  \url{http://www.ncbi.nlm.nih.gov/pubmed}

\bibitem{web2py}
  \emph{web2py}
  \url{http://www.web2py.com/}

\bibitem{googleApi}
  \emph{googleApi}
  \url{https://code.google.com/apis/}

\bibitem{mediaWiki}
  \emph{MediaWiki API}
  \url{http://www.mediawiki.org/wiki/API:Main_page}

\end{thebibliography}

\end{document}
